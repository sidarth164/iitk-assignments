\documentclass[12pt,a4paper]{article}
\usepackage[top=20pt, left=60pt, right=60pt]{geometry}
\usepackage[]{dirtytalk}
\usepackage[utf8]{inputenc}
\newcommand\blfootnote[1]{%
  \begingroup
  \renewcommand\thefootnote{}\footnote{#1}%
  \addtocounter{footnote}{-1}%
  \endgroup
}


\title{\textbf{ENG124\\Assignment 2}}
\author{Siddharth Agrawal 150716}
\date{March 30, 2017}

\begin{document}

\maketitle

We wonder many times - what is monolingualism or bilingualism? Is it simply knowing how to use language(s)? If we are multilingual, then does it mean we need to be exceptionally proficient in all languages - or - just understanding spoken words without the knowledge of writing is sufficient? Also does monolingualism truly mean speaking in only one way without any variation or does it include different versions for different times and places?
\\

Today I will ponder on above questions with respect to my usage of languages in my day to day life. I, Siddharth Agrawal, am a native Hindi speaker and also know English language. Now, it is clear that I am a bilingual. But do these two languages I know are used separately- exclusive of each other? The answer is \textbf{NO}. My brain naturally picks words from the two languages which are easier to use. For example - if I meet my friend after a long time, I'll say - \say{Hi Ram! Haven't seen you in a while. \textit{Kaise Ho?}}. Thus language usage is not simply `$1+1=2$' rather a blend of degrees of proficiencies of different languages which compliment each other.
\\

The language I use in academics is an extremely formal variant which focuses on writing. Now, the language used here is the standard textbook variant. The best example of it is that I am writing this article in standard English for academic purpose. Similarly, if I have to use Hindi, then I need to use its standard textbook format.
\\
\blfootnote{Italiced words are in Hindi.}

Social interactions with different people requires me to use different variations of Hindi and English. These depend on the degree of formality, seniority(to whom you are speaking to) etc. My way of speaking to my friends, family, neighbours, acquaintances and strangers are different. While speaking to my friends, I use the most casual and \textit{slang-filled} tone (mostly Hindi filled with English words) like, \say{\textit{Aur saale, Kaisa hai}}. With my family members, the tone is casual but respectful (again mostly Hindi). For example while speaking to my parents or other senior family members - I use the respectful addressing word like \textit{aap,aayie,etc.} while speaking to junior family members(like my brother), I don't use such respectful words but there is a care to minimise the usage of slang. While conversing with neighbours, there is a certain degree of formality depending on how close I am to him. If the neighbour is very close to me, then naturally the degree of formality reduces. Conversely, if I rarely meet that neighbour, automatically I start talking to him in the most polite tones. In all the above cases, I tend to use more Hindi and minimal English. But in front of people like my teachers, doctors, officers, etc. I incline towards using formal English as a mode of communication. The same applies towards strangers who seem to be well-off. For example I am travelling in a train and I wish to strike a conversation with the person next to me, then I use the most formal English I know like, "Hello, sir. My name is Siddharth Agrawal, May I know your good name" or "Excuse me, can I borrow your flashlight?" etc.
\\

For official communication, extremely formal approach is required. I tend to majorly use English in such cases. For example, while meeting with a professor, I would address him as \textit{Sir} and would talk to him in a polite and subservient manner. Similarly, while writing written official documents, I take special care in the tone and the words used, careful not to offend the other person. For example, while writing an official letter, I would use words like 'would', 'could' instead of 'will', 'can' to present myself in a humble manner.
\\

Generally, while writing, I use complete English (without mixing it with Hindi). But while using facebook, whatsapp - I use 'Hinglih' (Just the way I speak normally - mixing both Hindi and English) and incomplete sentences. For example - in whatsapp my conversations(with my mother) somewhat go like -\\
\renewcommand{\thefootnote}{$\star$} 
\say{hi Mom}\\
\say{\textit{kaise ho beta}}\\
\say{Fine.\textit{accha hoon}}\\
\say{eaten\footnote{note :- eaten is not a valid word still it is used here} food??}\\
\say{haan}\\
In contrast, while writing emails, I tend to use a more formal tone.
\\

When meeting and talking to strangers, my tone and degree of respect depends on the social status off the person to whom I am talking to. Like, when talking to a peon or a rickshaw-driver, my tone is authoritative (albeit respectful) and Hindi is used as mode of communication. If the stranger is a waiter at a 5-star hotel, or a call centre person etc, although I still maintain the authoritative tone byt the language used changes to \textit{mostly} English. If the stranger is a high ranking official or a professor at my university, my tone becomes subservient and English is used as a means to communicate. The mode of addressal in above three cases also differs. Usually the peon is referred to as '\textit{Bhaiya}', the waiter and the call centre person are referred to as '\textit{mister}' or '\textit{waiter}' or their names, while the official and professor ar referred to as '\textit{Sir}'.
\\

In cases where I write some articles or essays, I try to embellish my writing by using metaphors, similes, puns and flowery \& uncommon words. This type of writing - I neither use it in my day to day conversations nor do I use it for academic or official purposes. Also English is the language in which I am comfortable with while reading book, newspaper,etc. But I don't have any such preference for mass media.
\\

So, from the above examples it can be said that my mind automatically adjusts my tone and mode of language according to various different situations. Also, it can be verified that a language is not just limited to only one version as it changes in response to different different situations and also blends in with other languages to help the user better express himself. 

\blfootnote{Italiced words are in Hindi.}
\end{document}
